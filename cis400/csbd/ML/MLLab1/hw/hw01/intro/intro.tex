\documentclass{article}

\title{An introductioni to LaTeX and Me}
\author{Xiaozhi Li}
\date{\today}

\begin{document}

\maketitle{}

\begin{abstract}
  This is a brief introduction of myself using LaTex.This is my first
  time using LaTex. In this brief document I will indroduce where I am
  from, what my major is, and my interests in this course.
\end{abstract}
\textbf{Acknowledgments}: Thanks for Prof. Chin for teaching CIS
400-Certified Security By Design. This course taught me how to use
LaTex.

\section{Where I Grew Up}
\label{sec:where-i-grew}
\begin{description}
\item[My home town] I was born in Panzhihua, a city built on mountains
  for mining and smelting. My family moved to Chengdu when I was 8
  years old. Chengdu is known for it's humid weather, beautiful girls,
  and delicious food.
\item[What I liked] Both Panzhihua and Chengdu were cities located in
  Sichuan province, a state also known as Szechuan. Recently there is
  a meme on the internet from the famous cartoon show 'rick and morty'
  about how delicious a Mcdonald sauce was. That sauce was only
  available back in 2008 marketing for the 'Mulan' movie, and
  apparently casters from the cartoon wanted it so bad that they
  yelled for remake. Lucky for me because I was raised in Szechuan, I
  have little desire for a fake American comercialised fake version of
  Szechuan sauce. I have tasted all kinds of sauces, and peppers. My
  favorite kind is a small pepper chopped and stored in cooking
  oil. One drop of that is spicy enough to make a bow of food tastes
  wonderful. I love the food of my home town and am quite proud of it.
\item[What I disliked] I didn't like the weather in Chengdu, it is hot
  and humid in summer. From April to August the weather turns the
  entire city into a giant Sauna room. I always wonder why nobody was
  steamed to death during those years.

\end{description}
\section{My Program Of Study}
\label{sec:my-program-study}
\begin{description}
\item[Computer Science] I am currently a Computer Information Science
  student at ECS
\end{description}

\section{Why I Picked My Program of Study}
\label{sec:why-i-picked}
\begin{description}
\item[Career] I always wanted to be a game designer. I evaluated all the
  possible route for me to get the top of my possible career. Learning
  Computer Science was the best choice. My math and logic is better
  than my art skills, and learning how to programming can be really
  handy for me to evaluate the scope of my games. I later found out CE
  could have been a better option for me since I am not really
  interested in pursuing scientific math studies for computer, but I
  am already a senior and life has taught me these things do not
  matter in the end.
\end{description}



\section{Why I Am Taking CSBD}
\label{sec:why-i-am}
\begin{itemize}
\item This class is easy to follow, in a mechanical
  way--I can tell what to do and am capable of doing them in order to
  succeed, and doing that requires no more brain power than actually
  following a procedure. I can write an entire paper about my personal
  experience with memorizing informations while being efficient and
  relaxed. I believe there are quite a lot articles about that
  too. Prof. Susan Older once said it is the best of a student's
  interest to put in minimum effort and get best outcome when taking
  college courses. So the very same day I picked this course and
  dropped hers. I was in Prof. Chin's CIS 487 by program
  requirement. Then I noticed his special way of teaching. Prof. Chin
  uses online videos to demonstrate detailed materials, small quiz at
  end of the video to help remind us the notes, and he uses class to
  demonstrate crucial details while assuming students know very little
  about the books. I am a very experienced course taker, my nose tells
  me Prof. Chins course will be very easy to follow, and here I am.
\item My other candidate at this hour was CIS 554, a c++ course, yet I
  took CSE283 with the same professor by accident last semester, and
  he said both course teaches the same material and he doesn't want me
  retake it and sleep through. This course is one of his
  recommendations and quite fits my schedule.
\item I once dropped out of college because of familly conditions, and
  my grades were so bad, when I back to applying schools no
  engineering school would have me. I spent a year to make good grades
  at a small university before came to SU. Now I am 26 year-old and my
  both parents are not in labor force. I plan to make a living by next
  year when I graduate, yet there are zilions to learn. Learning
  modules of this course can save me a ton of work.

\item I always wanted to use ubuntu, I have download and used it before several times, but it either doesn't go well with my graphic card, or I do not know what to do once installed the system. The ubuntu VM just makes me happy.

\end{itemize}
\section{What I Hope to Learn in CSBD}
\label{sec:what-i-hope}
\begin{enumerate}
\item I believe any knowledge, especially the ones taught at college,
are precious and valuable work like the most advanced cuisine by
chefs. Hence I do not eat them just by taste. I hope I learn what is
meant to be taught, and I hope my stomach can be filled.
\item I was told using latex and writing reports in this class gives
e good resume that outstands some graduate student standards. Yes,please.
\item I am not good at semantics and logic signs, but as a CS student
  I need to be. This course could help me with that.
\item I am here to take the credits, and earn a B+, or.. an A, no,
  really I want an A. What do I have to learn to get an A? I hope to
  learn that.
\end{enumerate}
\end{document}
