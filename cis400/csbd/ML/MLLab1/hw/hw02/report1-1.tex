%% ---------------------------------------------------
%% this is a report using latex and ML
%% this is the work of Xiaozhi Li for class CIS 400 CSBD
%% this file is writen under the guidence of the course, which is taught
%% by Professor Shiu-Kai Chin

%% 2017/9/13
%% ---------------------------------------------------
\documentclass{report}
\title{Report for Simple ML Example}
\author{\textbf{Xiaozhi Li}}
\date{\textbf{13 September 2017}}



%% Formats:
%% ---------------------------------------------------
%% 634format specifies the format of our reports
%% ---------------------------------------------------
\usepackage{634format}

%% ---------------------------------------------------
%% enumerate 
%% ---------------------------------------------------
\usepackage{enumerate}

%% ---------------------------------------------------
%% listings is used for including our source code in reports
%% ---------------------------------------------------
\usepackage{listings}
\usepackage{textcomp}

%% ---------------------------------------------------
%% Packages for math environments
%% ---------------------------------------------------
\usepackage{amsmath}

%% ---------------------------------------------------
%% Packages for URLs and hotlinks in the table of contents
%% and symbolic cross references using \ref
%% ---------------------------------------------------
\usepackage{hyperref}

%% ---------------------------------------------------
%% Packages for using HOL-generated macros and displays
%% ---------------------------------------------------
%% end of formats

\usepackage{holtex}
\usepackage{holtexbasic}
\input{commands}

\begin{document}
%% --------------------------------------------------- the listings
%% parameter "language" is set to "ML"
%% ---------------------------------------------------
\lstset{language=ML}


\maketitle{}

\begin{abstract}
  The project brings together elements of functional programming in ML
  and documentation ins \LaTeX{}.  The purpose of this project is to
  lay the groundwork for credibility: results that are thoroughly
  documented and easily reproducible by independent third parties. We
  establish the documentation and programming infrastructure where
  each chapter documents a problem or exercise.  Within each chapter
  are sections stating or showing:
  \begin{itemize}
  \item Problem statement
  \item Relevant code
  \item Test results
  \end{itemize}

  For each problem or exercise-oriented chapter in the main body of
  the report is a corresponding chapter in the Appendix containing the
  source code in ML.  This source code is not pasted into the
  Appendix.  Rather, it is input directly from the source code file
  itself. This means changes in source code are easily captured in the
  report by recompiling the report in \LaTeX{}.

We introduce the use of style files and packages. Specifically, we use:
  \begin{itemize}
  \item a style file for the course, \emph{634format.sty}, 
  \item the \emph{listings} package for displaying and inputting ML
    source code, and
  \item HOL style files and commands to display interactive ML/HOL
    sessions.
  \end{itemize}

  Finally, we show how to:
  \begin{itemize}
  \item easily generate a table of contents for the report, and
  \item refer to chapter and section labels in our report.
  \end{itemize}

  There are numerous \LaTeX{} tutorials on the web, for example,
  \url{https://www.latex-tutorial.com}, is very accessible for
  beginners.

 \end{abstract}
\begin{acknowledgments}
  We gratefully acknowledge the hard work, trust, and dedication of
  our past students in the Syracuse University Cyber Engineering
  Semester and the Air Force Research Laboratory's Advanced Course
  (ACE) in Engineering Cybersecurity Boot Camp.  They bridged dreams
  and reality.  In addition, this report is generated following the
  guidence of the course CIS 400-CSBD taught by Professor Shiu-kai
  Chin.
\end{acknowledgments}

\tableofcontents{}

\chapter{Executive Summary}
\label{cha:executive-summary-1}

\textbf{All requirements for this project are satisfied}.
Specifically,
\begin{description}
\item[Report Contents] \ \\
  Our report has the following content:
  Our report has the following content:
  \begin{enumerate}[{}]
  \item Chapter~\ref{cha:executive-summary-1}: Executive Summary
  \item Chapter~\ref{cha:exercise-2.5.1}: Exercise 2.5.1
    \begin{enumerate}[{}]
    \item Section~\ref{sec:problem-statement-5}: Problem Statement
    \item Section~\ref{sec:relevant-code}: Relevant code
    \item Section~\ref{sec:test-cases}: Test results
    \end{enumerate}

    \item Chapter~\ref{cha:exercise-3.4.1}
    \begin{enumerate}[{}]
    \item Section~\ref{sec:problem-statement-6}: Problem Statement
    \item Section~\ref{sec:relevant-code-3}: Relevant code
    \item Section~\ref{sec:test-result}: Test results
    \end{enumerate}


  \item Chapter~\ref{cha:exercise-3.4.2}
   \begin{enumerate}[{}]
   \item Section~\ref{sec:problem-statement-7}
   \item Section~\ref{sec:relevant-code-3}
   \item Section~\ref{sec:test-cases-1}
     \item Section~\ref{sec:explain-error}
   \end{enumerate}
 \item Chapter~\ref{cha:appendix-a:-exercise}:Appendix A: Exercise
   2.5.1 Source Cod
  \item Chapter~\ref{cha:appendix-b:-exercise}:Appendix B: Exercise 3.4.1 Source Cod
  \item Chapter~\ref{cha:appendix-c:-exercise}:Appendix C: Exercise 3.4.2 Source Cod

  \end{enumerate}
\item[Reproducibility in ML and \LaTeX{}] \ \\
  Our ML and \LaTeX{} source files compile with no errors.
\end{description}

\chapter{Exercise 2.5.1}
\label{cha:exercise-2.5.1}

\section{Problem Statement}
\label{sec:problem-statement-5}

In this exercise we are to define in ML the following functions:
\begin{align*}
  timesPlus\; x\; y &= (x*y,\; x+y)\\
\end{align*}

\section{Relevant Code}
\label{sec:relevant-code}

The following code takes advantage of function definition using
\emph{fun} in ML, and \emph{currying}, i.e., defining functions with multiple arguments as a sequence of functions. This supports partial evaluation.

\lstset{frameround=fftt}
\begin{lstlisting}[frame=tRBL]
  fun timesPlus x y= (x*y, x+y);
\end{lstlisting}

\section{Test Cases}
\label{sec:test-cases}
The required test cases for \emph{timesPlus} are as follows.
\begin{lstlisting}[frame = TB]
(****************************************************************)
(* Test Cases Specified in the requirements                     *)
(****************************************************************)
timesPlus 100 27;
timesPlus 10 26;
timesPlus 1 25;
timesPlus 2 24;
timesPlus 30 23;
timesPlus 50 200;

\end{lstlisting}

\section{Test Results}
\label{sec:test-results}

\setcounter{sessioncount}{0}
\begin{session}
  \begin{scriptsize}
\begin{verbatim}

> > > > val ListA = [(0, "Alice"), (1, "Bob"), (3, "Carol"), (4, "Dan")]:
   (int * string) list
> val ListB = [(1, "Bob"), (3, "Carol"), (4, "Dan")]: (int * string) list
val elB = (0, "Alice"): int * string
> val elc1 = 0: int
val elc2 = "Alice": string
> > val elc3 = (1, "Bob"): int * string
val elc4 = (3, "Carol"): int * string
val elc5 = (4, "Dan"): int * string
> 
\end{verbatim}
  \end{scriptsize}
\end{session}

\chapter{Exercise 3.4.1}
\label{cha:exercise-3.4.1}

\section{Problem Statement}
\label{sec:problem-statement-6}
In this exercise we are to define in ML the following functions:
\begin{align*}
  &val\; ListA\; =\; [(0,\; "Alice"),\; (1,\;"Bob"),\; (3,\; "Carol"),\;(4,\;"Dan")];\\
                      &val\; elB:: ListB\;= \;ListA;\\
                      &val\; (elc1,elc2)\;=\;elB;\\
                      &val\; [elc3,elc4,elc5]\;=\;ListB;\\
\end{align*}

\section{Relevant Code}
\label{sec:relevant-code-2}


The following code takes advantage of function definition using
\emph{fun} in ML, and \emph{currying}, i.e., defining functions with multiple arguments as a sequence of functions. This supports partial evaluation.

\lstset{frameround=fftt}
\begin{lstlisting}[frame=tRBL]

 val ListA = [(0, "Alice"), (1,"Bob"), (3, "Carol"),(4,"Dan")];
 val elB:: ListB= ListA;
 val (elc1,elc2)=elB;
 val [elc3,elc4,elc5]=ListB;
\end{lstlisting}


\section{Test Result}
\label{sec:test-result}
\setcounter{sessioncount}{0}
\begin{session}
  \begin{scriptsize}
\begin{verbatim}

> > > > val ListA = [(0, "Alice"), (1, "Bob"), (3, "Carol"), (4, "Dan")]:
   (int * string) list
> val ListB = [(1, "Bob"), (3, "Carol"), (4, "Dan")]: (int * string) list
val elB = (0, "Alice"): int * string
> val elc1 = 0: int
val elc2 = "Alice": string
> > val elc3 = (1, "Bob"): int * string
val elc4 = (3, "Carol"): int * string
val elc5 = (4, "Dan"): int * string
> 
\end{verbatim}
  \end{scriptsize}
\end{session}



\chapter{Exercise 3.4.2}
\label{cha:exercise-3.4.2}


\section{Problem Statement}
\label{sec:problem-statement-7}
In this exercise we are to define in ML the following functions:
\begin{align*}
&val\; (x1,x2,x3)\; = \;(1,\;true,\;"Alice");\\
&val\; pair1\; =\; (x1,\;x3);\\
&val\; list1\; =\; [0,\;x1,\;2];\\
&val\; list2\; =\; [x2,\;x1];\\
&val\; list3\; =\; (1\; ::\; [x3]);\\
\end{align*}


\section{Relevant Code}
\label{sec:relevant-code-3}
The following code takes advantage of function definition using
\emph{fun} in ML.

\lstset{frameround=fftt}
\begin{lstlisting}[frame=tRBL]

val (x1,x2,x3) = (1,true,"Alice");
val pair1 = (x1,x3);
val list1 = [0,x1,2];
val list2 = [x2,x1];
val list3 = (1 :: [x3]);
val ListA = [(0, "Alice"), (1,"Bob"), (3, "Carol"),(4,"Dan")];
val elB:: ListB= ListA;
val (elc1,elc2)=elB;
val [elc3,elc4,elc5]=ListB;

\end{lstlisting}
\section{Test Cases}
\label{sec:test-cases-1}

The following are the test results
\setcounter{sessioncount}{0}
\begin{session}
  \begin{scriptsize}
\begin{verbatim}

> > > >val x1 = 1: int
val x2 = true: bool
val x3 = "Alice": string
> val pair1 = (1, "Alice"): int * string
> val list1 = [0, 1, 2]: int list
> poly: : error: Elements in a list have different types.
   Item 1: x2 : bool
   Item 2: x1 : int
   Reason:
      Can't unify bool (*In Basis*) with int (*In Basis*)
         (Different type constructors)
Found near [x2, x1]
Static Errors
> poly: : error: Type error in function application.
   Function: :: : int * int list -> int list
   Argument: (1, [x3]) : int * string list   Reason:
      Can't unify int (*In Basis*) with string (*In Basis*)
         (Different type constructors)
Found near (1 :: [x3])
Static Errors
> > > > val ListA = [(0, "Alice"), (1, "Bob"), (3, "Carol"), (4, "Dan")]:
   (int * string) list
> val ListB = [(1, "Bob"), (3, "Carol"), (4, "Dan")]: (int * string) list
val elB = (0, "Alice"): int * string
> val elc1 = 0: int
val elc2 = "Alice": string
> > val elc3 = (1, "Bob"): int * string
val elc4 = (3, "Carol"): int * string
val elc5 = (4, "Dan"): int * string
> 
\end{verbatim}
  \end{scriptsize}
\end{session}

\subsection{Explain of error}
\label{sec:explain-error}
The errors occured in 3.4.2 are because of type matching. Errors in val list2 and val lsit3, is that list2 referenced x2 from (x1,x2,x3) = (1,true,"Alice"). Where x2 is the type bool, and HOL can't put bool and numbers in one list. List3 contains x3 which is a string type and Hol will not put string type and int in one list; therefore we got typeerrors.

\chapter{Appendix A: Exercise 2.5.1 Source Code}
\label{cha:appendix-a:-exercise}
The following code is from \emph{ex-2-5-1.sml}
\lstinputlisting{ML/ex-2-5-1.sml}

\chapter{Appendix B: Exercise 3.4.1 Source Code}
\label{cha:appendix-b:-exercise}
The following code is from \emph{ex-3-4-1.sml}
\lstinputlisting{ML/ex-3-4-1.sml}

\chapter{Appendix C: Exercise 3.4.2 Source Code}
\label{cha:appendix-c:-exercise}
The following code is from \emph{ex-3-4-2.sml}
\lstinputlisting{ML/ex-3-4-2.sml}





\end{document}